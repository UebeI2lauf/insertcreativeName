%\documentclass[12pt, a4paper]{article}
\documentclass[runningheads]{llncs}
\usepackage{graphicx}
\usepackage{wrapfig}
\usepackage{subfigure}
\usepackage{multirow}
\usepackage{hyperref}
\usepackage{amsmath}
\usepackage{amssymb}
\usepackage{capt-of}
\usepackage{multicol}
\usepackage{aliascnt}
% \usepackage{ngerman}
\usepackage[ansinew]{inputenc}
\usepackage[left=2.5cm,top=2.5cm]{geometry}

%\renewcommand\UrlFont{\color{blue}\rmfamily}

\begin{document}

\title{The effect of common network problems on students academic performance in an elearning-Environment
\thanks{Supported by Goethe University Frankfurt a. Main}}

\author{Lucas Laub\inst{1}\orcidID{6621331} \and
Alexander Perekhrest \inst{2}\orcidID{6379748}}

\institute{Goethe University Frankfurt, 60323 Frankfurt a. Main, Germany.}

\maketitle


\begin{abstract}
In the current light of the pandemic the worldwide use
of eLearning-Software experienced an unpresented boom.
We state the question how common network problems influence
the academic performance in an eLearning-Environment.
To provide answers an online questionnaire with deliberate
technical difficulties was constructed. Evaluating the performance
of the test and control group did not show any significant
differences.
\keywords{eLearning  \and Online-Learning \and academic performance.}
\end{abstract}

\section{Introduction}
When trying to transfer an already existing method on a relatively new platform, 
it's important to know the things that come with being on such a platform and the 
possible influences those things might have on the method.\\

In day-to-day usage of online platforms and services it's not uncommon to face some 
issues, whether it's execution, connectivity and so on. E-Learning-platforms are not 
particularly different to those. Therefore, we want to discuss, in this paper, to 
which extent these problems can influence the test-results of being on such an 
'issue-infected' platform in contrast to a well running platform with no issues.\\

We focused on common network issues.Which are defined by HTTP-status-codes, like 400 (Bad Request), 401 
(Unauthorized), 403 (Forbidden), 404 (Not Found), 408 (Request Timeout), as mostly being 'client-errors'.
\section{Materials and Methods}
\subsection{Summary}
To which extend do common networks problems in an eLearning-Environment
influence the academic performance of students. This information can be used to
re-evaluate the eLearning-Environment. 

\subsection{Participants}
The participants are students of the end of the 4 grade and consist of two groups the control group [CG] and
the test group [TG]. Each group is made up by 50 girls and 50 boys for a total of 200 participants.
It should be ensured that both groups prior to the experiment perform academically similar, if not a
comparison post experiment will be difficult. Students of the end of the 4th grade have the benefit of the
already finished primary school. Which ensures experience in simple problem solving and
reading comprehension. Furthermore in 4th grade we can still observe all academic capabilities
since the division of students happens in 5th grade. Also primary school has the least differences
in the curriculum between the federal stats.


\subsection{Preparations}
The experiment was conducted by creating a software implementing Fig.~\ref{fig1}.
This software\cite{ref_soft} allowed the tracking of {\itshape technical problems} introduced by
the software itself as well as the points and answers scored by each participant.
A maximum of 100 points could be scored, these were needed to evaluate the performance of
the students. Anymore name, gender, age, email, username and password were collected.
These parameters were used to identify students and for further analysis.
Other solutions which involved already existing software-suits were abandon due to lack
of customization.Additionally a room with an adequate number of computers with a fiber-connection
to the server are needed, to rule out uncontrolled network problems.
Half of the computers are manipulated and simulate the network problems with the
use of the software.

\subsection{Procedure}
The participants are welcomed and thanked for their time.
The students were randomly selected for either the control group or the test group
prospective known as the error group. Then the students where placed in front of a computer
which was either manipulated or untouched based on their group.They then create an account
and start solving the questions.
After they finished all questions they log out.
The students are rewarded with cookies and again thanked for their time.

\begin{figure}
    \includegraphics[width=\textwidth]{UML Prototyp.PNG}
    \caption{A logic flow chart , representing how an implementation could operate.
    The black circle is the user interacting with the software. The website would
    consist of two parts. A front-end handling user interaction and the creation of {\itshape bugs}.
    The back-end responsible for saving the collected data and ensuring the front-end
    remains operational.} \label{fig1}
\end{figure}
\newpage
\section{Results}
The results are displayed in chronological order of the analysis, there is
no emphasis on the significance given by the order itself. The following data is
what we would expect in an actual experiment. To recreate the data see \cite{ref_soft}.

\subsection{Control Group(50f/50m) vs Error Group(50f/50m)}
\begin{figure}[!h]
    \begin{minipage}{0.4\textwidth}        
        \includegraphics[width=\textwidth]{code/generate/all.png}
        \caption{The colored squares in the box-plot displays
        the upper and lower quartile of points earned by the control group (50f/50m) and
        the error group (50f/50m).The green line marks the mean of all data-points in the group.
        The gray line marks the median  of the given group.} \label{fig2}
    \end{minipage}
\hfill
\begin{minipage}{0.4\textwidth}
\captionof{table}{The calculated median, standard deviation and t, p-values
    for the control group (50f/50m) the error group (50f/50m).
    The t,p-values were calculated by using a two-sided t-test.}
\begin{tabular}[]{| c | c | c |}
        \hline
        & control group & error group \\
        \hline
        median & 53.0&43.5 \\
        \hline
        standard deviation & 29.278&29.826 \\
        \hline
        t-value & \multicolumn{2}{c|}{0.728} \\
        \hline
        p-value & \multicolumn{2}{c|}{0.467} \\
        \hline            
\end{tabular}
\end{minipage}
\end{figure}

\subsection{Control Group(50f) vs Error Group(50f)}
\begin{figure}[!ht]
    \begin{minipage}{0.4\textwidth}        
        \includegraphics[width=\textwidth]{code/generate/all.png}
        \caption{The colored squares in the box-plot displays
        the upper and lower quartile of points earned by the control group (50f) and
        the error group (50f).The green line marks the mean of all data-points in the group.
        The gray line marks the median  of the given group.} \label{fig3}
    \end{minipage}
\hfill
\begin{minipage}{0.4\textwidth}
\captionof{table}{The calculated median, standard deviation and t, p-values
    for the control group (50f) the error group (50f).
    The t,p-values were calculated by using a two-sided t-test.}
\begin{tabular}[]{| c | c | c |}
        \hline
        & control group & error group \\
        \hline
        median & 55.5&40.0 \\
        \hline
        standard deviation & 27.85&30.933 \\
        \hline
        t-value & \multicolumn{2}{c|}{0.949} \\
        \hline
        p-value & \multicolumn{2}{c|}{0.345} \\
        \hline            
\end{tabular}
\end{minipage}
\end{figure}

\subsection{Control Group(50m) vs Error Group(50m) }
\begin{figure}[!h]
    \begin{minipage}{0.43\textwidth}        
        \includegraphics[width=\textwidth]{code/generate/all.png}
        \caption{The colored squares in the box-plot displays
        the upper and lower quartile of points earned by the control group (50m) and
        the error group (50m).The green line marks the mean of all data-points in the group.
        The gray line marks the median  of the given group.} \label{fig4}
    \end{minipage}
\hfill
\begin{minipage}{0.43\textwidth}
\captionof{table}{The calculated median, standard deviation and t, p-values
    for the control group (50m) the error group (50m).
    The t,p-values were calculated by using a two-sided t-test.}
\begin{tabular}[]{| c | c | c |}
        \hline
        & control group & error group \\
        \hline
        median & 52.5&50.5 \\
        \hline
        standard deviation & 30.638&28.495 \\
        \hline
        t-value & \multicolumn{2}{c|}{0.08} \\
        \hline
        p-value & \multicolumn{2}{c|}{0.936} \\
        \hline            
\end{tabular}
\end{minipage}
\end{figure}

\subsection{Control Group(50f) vs Control Group(50m)}
\begin{figure}
    \begin{minipage}{0.43\textwidth}        
        \includegraphics[width=\textwidth]{code/generate/all.png}
        \caption{The colored squares in the box-plot displays
        the upper and lower quartile of points earned by the control group (50f) and
        the control group (50m).The green line marks the mean of all data-points in the group.
        The gray line marks the median  of the given group.} \label{fig5}
    \end{minipage}
\hfill
\begin{minipage}{0.43\textwidth}
\captionof{table}{The calculated median, standard deviation and t, p-values
    for the control group (50f) the control group (50m).
    The t,p-values were calculated by using a two-sided t-test.}
\begin{tabular}[]{| c | c | c |}
        \hline
        & control group(f) & control group(m) \\
        \hline
        median & 55.5&52.5 \\
        \hline
        standard deviation & 27.85&30.638 \\
        \hline
        t-value & \multicolumn{2}{c|}{0.101} \\
        \hline
        p-value & \multicolumn{2}{c|}{0.919} \\
        \hline            
\end{tabular}
\end{minipage}
\end{figure}
\clearpage
\subsection{Error Group(50f) vs Error Group(50m)}
\begin{figure}
    \begin{minipage}{0.43\textwidth}        
        \includegraphics[width=\textwidth]{code/generate/all.png}
        \caption{The colored squares in the box-plot displays
        the upper and lower quartile of points earned by the error group (50f) and
        the error group (50m).The green line marks the mean of all data-points in the group.
        The gray line marks the median  of the given group.} \label{fig6}
    \end{minipage}
\hfill
\begin{minipage}{0.43\textwidth}
\captionof{table}{The calculated median, standard deviation and t, p-values
    for the error group (50f) the error group (50m).
    The t,p-values were calculated by using a two-sided t-test.}
\begin{tabular}[]{| c | c | c |}
        \hline
        & error group & error group \\
        \hline
        median & 40.0&50.5 \\
        \hline
        standard deviation & 30.933&28.495 \\
        \hline
        t-value & \multicolumn{2}{c|}{- 0.759} \\
        \hline
        p-value & \multicolumn{2}{c|}{0.45} \\
        \hline            
\end{tabular}
\end{minipage}
\end{figure}

\subsection{Control Group(50f) vs Error Group(50m)}
\begin{figure}
    \begin{minipage}{0.43\textwidth}        
        \includegraphics[width=\textwidth]{code/generate/all.png}
        \caption{The colored squares in the box-plot displays
        the upper and lower quartile of points earned by the control group (50f) and
        the error group (50m).The green line marks the mean of all data-points in the group.
        The gray line marks the median  of the given group.} \label{fig7}
    \end{minipage}
\hfill
\begin{minipage}{0.43\textwidth}
\captionof{table}{The calculated median, standard deviation and t, p-values
    for the control group (50f) the error group (50m).
    The t,p-values were calculated by using a two-sided t-test.}
\begin{tabular}[]{| c | c | c |}
        \hline
        & control group & error group \\
        \hline
        median & 55.5&50.5 \\
        \hline
        standard deviation & 27.85&28.495 \\
        \hline
        t-value & \multicolumn{2}{c|}{0.19} \\
        \hline
        p-value & \multicolumn{2}{c|}{0.85} \\
        \hline            
\end{tabular}
\end{minipage}
\end{figure}
\clearpage
\subsection{Control Group(50m) vs Error Group(50f)}
\begin{figure}
    \begin{minipage}{0.43\textwidth}        
        \includegraphics[width=\textwidth]{code/generate/all.png}
        \caption{The colored squares in the box-plot displays
        the upper and lower quartile of points earned by the control group (50m) and
        the error group (50f).The green line marks the mean of all data-points in the group.
        The gray line marks the median  of the given group.} \label{fig8}
    \end{minipage}
\hfill
\begin{minipage}{0.43\textwidth}
\captionof{table}{The calculated median, standard deviation and t, p-values
    for the control group (50m) the error group (50f).
    The t,p-values were calculated by using a two-sided t-test.}
\begin{tabular}[]{| c | c | c |}
        \hline
        & control group & error group \\
        \hline
        median & 52.5&40.0 \\
        \hline
        standard deviation & 30.638&30.933 \\
        \hline
        t-value & \multicolumn{2}{c|}{0.81} \\
        \hline
        p-value & \multicolumn{2}{c|}{0.42} \\
        \hline            
\end{tabular}
\end{minipage}
\end{figure}

\section{Discussion}
The collected data is very clear. Network problems do not affect students academic performance
in a statistically firm manner.

The results can not be transposed on younger students.
Since younger students might be unable to read properly or lack experience with problem solving
in general. Furthermore can the data not be used to make an estimate for students with disabilities.
However the performance of older students should be similar to the tested students.

The study can make a general statement on network problems. However not all possible
problems are simulated. It is possible that a problem exists that does affect the academic performance of
students significantly. Also long-term effects can not be discovered by this study since it is very time limited.
Furthermore design-flaws of the eLearning-Environment are also overlooked or ignored.
These problems require additional studies.
\section{Conclusion}
In this paper we could clearly see the impact of platform-issues on the students performance in an 
exam-scenario and it is not significant. There might still be certain aspect of long-term-effects,
which we couldn't simulate due to our test setup, because there was a slight worsening happening 
when comparing our outcomes. Further studies should focus on the long-term effect of technical problems to deepen our understanding of the technology.
\section{References}
\begin{thebibliography}{8}
    \bibitem{ref_soft}
    Github, \url{https://github.com/UebeI2lauf/insertcreativeName/tree/main/code}. Last accessed 8 Jul 2021
    \end{thebibliography}
\end{document}
