\documentclass[12pt, a4paper]{article}
%\documentclass[runningheads]{llncs}
\usepackage{graphicx}
\usepackage{wrapfig}
\usepackage{subfigure}
\usepackage{multirow}
\usepackage{hyperref}
\usepackage{amsmath}
\usepackage{amssymb}
% \usepackage{ngerman}
\usepackage[ansinew]{inputenc}
\usepackage[left=2.5cm,top=2.5cm]{geometry}
\usepackage[section]{placeins}

\renewcommand\UrlFont{\color{blue}\rmfamily}

\begin{document}

\title{The effect of commen network problems on students academic performance in an elearning-Environment \thanks{Supported by Goethe University Frankfurt a. Main}}

\author{Lucas Laub \inst{1}\orcidID{6621331} \and
Alexander Perekhrest \inst{2}\orcidID{6379748}}

\institute{Goethe University Frankfurt, 60323 Frankfurt a. Main, Germany.}

\maketitle

\begin{abstract}
In the current light of the pandemic the worldwide use
of eLearning-Software experienced an unpresented boom.
We state the question how common network problems influence
the academic performance in an eLearning-Environment.
To provide answers an online questionnaire with deliberate
technical difficulties was constructed. Evaluating the performance
of the test and control group did not show any significant
differences.
\keywords{eLearning  \and Online-Learning \and academic performance.}
\end{abstract}


\section{Introduction}
When trying to transfer an already existing method on a relatively new platform, it's important to know the things that come with being on such a platform and the possible influences those things might have on the method.\\\\
In day-to-day usage of online platforms and services it's not uncommon to face some issues, whether it's execution, connectivity and so on. E-Learning-platforms are not particularly different to those. Therefore, we want to discuss, in this paper, to which extent these problems can influence the test-results of being on such a 'issue-infected' platform in contrast to a well running platform with no issues.\\\\
We tried to focus on the most frequent issues we faced in our experience of browsing on different platforms, which are defined by HTTP-status-codes, like 400 (Bad Request), 401 (Unauthorized), 403 (Forbidden), 404 (Not Found), 408 (Request Timeout), as mostly being 'client-errors'.\\\\

\section{Materials and Methods}
\subsection*{Preparations}
The experiment was conducted by creating a software implementing Fig.~\ref{fig1}.
This software allowed the tracking of {\itshape technical problems} introduced by
the software itself as well as the points and answers scored by each participant.
Additionally a room with an adequate number of computers with a fiber-connection
to the server are needed, to rule out uncontrolled network problems.
Half of the computers are manipulated and simulate the network problems with the
use of the software.

\begin{figure}[h]
    \includegraphics[width=\textwidth]{UML Prototyp.PNG}
    \caption{A logic flow chart , representing how an implementation could operate.
    The black circle is the user interacting with the software. The website would
    consist of two parts. A front-end handling user interaction and the creation of {\itshape bugs}.
    The back-end responsible for saving the collected data and ensuring the front-end
    remains operational.} \label{fig1}
\end{figure}

\subsection*{Participants}
The participants are students of the 5th grade and consist of two groups the control group [CG] and
the test group [TG]. Each group is made up by 50 girls and 50 boys for a total of 200 participants.
It should be ensured that both groups prior to the experiment perform academically similar, if not a
comparison post experiment will be difficult.
\newpage
\section{Results}
\begin{figure}[h]
    \includegraphics[width=\textwidth]{results.PNG}
\end{figure}
\begin{center}
    \includegraphics[width=0.5\textwidth]{all.png}
\end{center}
\begin{figure}[h]
    \includegraphics[width=.5\textwidth]{female.png}
    \includegraphics[width=.5\textwidth]{male.png}
    \includegraphics[width=.5\textwidth]{fsav_msav.png}
    \includegraphics[width=.5\textwidth]{ferr_merr.png}
\end{figure}
\begin{figure}[h]
    \includegraphics[width=.5\textwidth]{fsav_merr.png}
    \includegraphics[width=.5\textwidth]{msav_ferr.png}
\end{figure}


\section{Discussion}
\section{Conclusion}
\section{References}
\end{document}
